\subsection{Tseitin transformation}\label{pre:tseitin}

Tseitin transformation is a procedure that is taking a propositional formula and compute a new formula in conjunctive normal form that is equisatisfiable to the initial formula. \stodo{should write its also linear in the size of the input.}

As stated in \cite{tseitin}, for each logical connective we introduce a new variable that is equivalent with its formula, and replace in the subformulas. These equivalences $T \leftrightarrow A \circ B$, where $\circ$ is a logical connective, can be transformed in a logical equivalent CNF, in table \ref{tab:tseitin} we display a portion of those CNF.

\begin{table}[H]
\center
\begin{tabular}{|c|c|}
    \hline
    Logical connective & Conjunctive normal form\\
    \hline
    $T \leftrightarrow A \land B$ & $(T \lor \overline{A} \lor \overline{B}) \land (\overline{T} \lor A) \land (\overline{T} \lor B)$ \\
    \hline
    $T \leftrightarrow A \lor B$ & $(\overline{T} \lor A \lor B) \land (T \lor \overline{A}) \land (T \lor \overline{B})$ \\
    \hline
\end{tabular}
\caption{Tseitin transformation of a logical connective into its conjunctive normal form.}
\label{tab:tseitin}
\end{table}

This transformation works as follows: given a propositional formula $\phi$, for each subformula we introduce a Tseitin variable and add this cnf to $\phi_T$, after we introduce all the subformulas, $\phi$ is equisatisfiable with $\phi_T \land T_0$, where $T_0$ corresponds to the first logical connective of $\phi$.

We illustrate this transformation in the next example:

\begin{example}
    Let's have the formula $(A \land B) \lor C$, first we should point that, this is not in CNF. We introduce a Tseitin variable for each subformula $T_1 \leftrightarrow (A \land B)$ and $T_0 \leftrightarrow (T_1 \lor C)$. We write the conjunctive normal form for each equivalence $\varphi_T = (T_1 \lor \overline{A} \lor \overline{B}) \land (\overline{T_1} \lor A) \land (\overline{T_1} \lor B) \and (\overline{T_0} \lor T_1 \lor C) \land (T_0 \lor \overline{T_1}) \land (T_0 \lor \overline{C})$. Finally, $\varphi_T \land T_0$ is the transformation of the initial formula in CNF.
\end{example}

A sketch of why it is working, and why we need to add the last variable to the formula is that each of those introduce Tseitin variable just takes the value of their equivalence, and the need for the conjunction with $T_0$ is that we want our end formula to be true, without it, we can just set all Tseitin variable to true, and do not use anything else, and get a true formula. And a formal proof of this is that, given $\varphi$ and $\varphi_T \land T_0$, if $\varphi$ is satisfiable, then using the assignment we can set our $T_i$ accordingly, and $T_0$ is true because $\varphi$ is true to be satisfiable, in case $\varphi$ is unsatisfiable, suppose $\varphi_T \land T_0$ is satisfiable, but this cannot be the case, because we can get an assignment that should be true for $\varphi$ thus $\varphi_T \land T_0$ is also unsatisfiable.

With the transformation for propositional formula in a CNF in place, in a QBF PCNF setting we are raising the question of where we put the Tseitin variables in the prefix. If we do not assign them in and quantified block, they will be treated as free variables, thus in the outermost existential block (if the first block is universal, we just add another existential block outside). But this has the following counter-example $\forall x (x \land \overline{x})$ that is true to $\exists t \forall x (t \lor \overline{x} \lor x) \land (\overline{t} \lor x) \land (\overline{t} \lor \overline{x}) \land t$ which is false. Thus, we can form the following claim:

\begin{claim}
    Given a QBF of form $\Pi\varphi$, where $\varphi$ is quantifier free but is not in conjunctive normal form. $\Pi\varphi$ is equisatisfiable with $\Pi \exists T (\varphi_T \land T_0)$, where $T$ is the set of Tseitin variables, $\varphi_T$ is the Tseitin transformation of subformulas and $T_0$ is the first logical connective that is applied to the formula. (If last quantified block of $\Pi$ is $\exists$, then $T$ is appended to it.)
\end{claim}

\begin{proof}
    Similar with the propositional case, by applying the semantic definition \ref{def:semantics} of QBF, we end up with a matrix that is formed prefixed only by existential block of Tseitin variables. Due to the prefix being only existential we can use the reasoning at propositional level, and we can use the prior sketch proof.
\end{proof}