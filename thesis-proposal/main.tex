\documentclass[12pt, a4paper]{article}
\usepackage[margin=2cm]{geometry}
\usepackage{xcolor}
\usepackage{todonotes}
\usepackage[colorlinks=true, 
            linkcolor=blue,
            citecolor=green,
            pdftitle={Master Thesis Proposal}]{hyperref}

\newcommand{\textrosu}[1]{\textcolor{red}{#1}}

\title{
    \textbf{Master Thesis Proposal}\\
    Certified Circuit Reconstruction for QBF
}
\author{Student: Mihai-Alexandru Weng \\ Advisor: Dr. Friedrich Slivovsky}
\date{\today}

\setlength{\parindent}{0pt}
\linespread{1.25}

\begin{document}

\maketitle
\tableofcontents
\newpage

\section{Motivation \& Problem Statement}



% Given the success of SAT solvers, and its applications, we can go one step further and generalize the algorithms for the quantified version of boolean formulas, \cite{biere_chapter_2021}. The problems' domain for the QBF version are founded in PSPACE class, whilst SAT problems are in NP class. QBFs can encode applications ranging from verification, model checking, to Artificial intelligence, games strategies, \cite{shukla_survey_2019}.

% The current methods and tools used in certifying proofs revolve around usage of a trusted program (usually proved with formal methods). This program often introduce a custom proof format that facilitate the code running, \cite{de_moura_efficient_2017}, \cite{bryant_certified_2023}.

% Translation of CNF to non-CNF format, \cite{jordan_non-cnf_nodate}, leaves room for improvement by trying to validate formally din transformation.

\newpage

\section{Aim of the Work}

The objective of the thesis is to introduce an additional way of generating a QRAT proof for a given QBF in QDIMACS format, using its circuit translation.

In this way, we can achieve extra assurance for the circuit translation by checking the produce proof can also solve the input before translation. And also, utilize the favorable input for the QBF solver, based on empirical result \cite{jordan_non-cnf_nodate}, a circuit based format.

\newpage

\section{Methodological Approach}
\begin{enumerate}
    \item Better understanding of the tools involved in the process.
    \item Devise test samples for checking various edge cases.
    \item Automate the workflow.
    \item Modify a QCIR QBF (QCDCL) solver to output the CNF encoding.
    \item Implement conversion from Q-resolution proof to QRAT.
    \item Implement a program for getting a (Q)RAT derivation of the CNF encoding using a SAT solver.
    \item Verify against the test sample.
    \item Using common benchmarks for testing.
\end{enumerate}

\newpage

\section{Structure of the Work}

\begin{enumerate}
    \item Introduction
    \item Preliminaries
    \begin{enumerate}
        \item Q-resolution proof
        \item QRAT proof
        \item QBF input formats (CNF / non-CNF)
    \end{enumerate}
    \item Extension of circuit based solver
    \item QRAT proof for QDIMACS from QCIR
    \item Testing
    \item Conclusion
\end{enumerate}

\newpage

\section{State-of-the-Art}

In \cite{klieber_formal_nodate}, it is raised the problem that simple Tseitin translation can be harmful to the QBF, a CNF QBF solver could take exponential time for a trivial input before the translation. Thus, in the previous work, a tool for transforming a QDIMACS into circuit form is developed. Also, in \cite{jordan_non-cnf_nodate} testing the circuit format, QCIR \cite{noauthor_qcir-g14_nodate}, is noted the direction of the ongoing research for non-CNF solvers in improving the translation and certification. In plus, we can see an interest in developing those solvers for circuits format for competition. This way, it can open the possibility for combining the best of both worlds solvers, CNF and non-CNF, \cite{beyersdorff_circuit-based_2018}.

\newpage

\section{Relevance to the Curricula of Logic and Computation}

The topic of certifying proof, in development of validation tools for boolean formulas, touches many areas covered in the Logic and Computation syllabus. Those branches are: algorithms and data structures, logic, formal verification, complexity. Courses relevant to the thesis' content are:

\begin{itemize}
    \item 186.814 Algorithmics
    \item 186.182 Seminar on Algorithms
    \item 184.090 SAT Solving
    \item 181.145 Computer Aided Verification
    \item 185.291 Formal Methods in Computer Science
    \item 185.A45 Logic and Computability
    \item 184.068 Artificial Intelligence Seminar
\end{itemize}

\newpage

\bibliographystyle{ieeetr}
\bibliography{QBF}

\end{document}